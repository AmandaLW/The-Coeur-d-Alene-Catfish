\documentclass[12pt]{article}
\usepackage[paperwidth=8.5in,paperheight=11in,margin=1in]{geometry}
\usepackage{float}
\usepackage{lipsum}
\usepackage{parskip}
\usepackage{bbding}
\usepackage{amssymb}
\usepackage{titlesec} 
\usepackage{graphicx}
\usepackage{hyperref}
\usepackage{setspace}
\usepackage[normalem]{ulem}
\usepackage[section]{placeins}
\usepackage[toc,page]{appendix}
\newcounter{subsubsubsection}[subsubsection]
\newcommand{\tline}{\hspace{-2.3pt}$\bullet$ \hspace{5pt}}
\hypersetup{colorlinks=true, linkcolor=black, urlcolor=blue}
\setlength{\parindent}{15pt} % Indent paragraphs (automatically)
\usepackage{pdfpages, caption}


\makeatother
\makeatletter
\setlength{\@fptop}{0pt}

\newcommand\tab[1][1cm]{\hspace*{#1}}

\definecolor{myRed}{RGB}{248, 0, 0}
\definecolor{myGreen}{RGB}{0, 208, 0} %full green too light -P
\definecolor{myBlue}{RGB}{0, 0, 248}

\begin{document}
	\begin{titlepage}
		\centering	
%		\vspace{.25cm}
    
    \begin{figure}[h]
      \centering
      \includegraphics[width=0.38\linewidth]{assets/New_Logov1.png}
    \end{figure} 
  
  {\huge\bfseries The Coeur d'Alene Catfish: \\ Research Portfolio\par}
    
    \title{}
    \date{\vspace{-5ex}} %blank date -P
    \author{%
    	\makebox[.3\linewidth]{\Large\itshape Adrian Beehner}\\Team Manager\\
    	\and \makebox[.3\linewidth]{\Large\itshape Samantha Freitas}\\Designer\\
    }
    \let\newpage\relax\maketitle %don't make a new page -P
    \maketitle		
    
    \vspace{1cm} 
    
    {\scshape\Large 
      May 2018 - August 2018 \\
      Autonomous Robotic Submarine Project\\ 
      Sponsors - Idaho Water Resources Research Institute, United States Geologic Survey, UofI \\
      Supervisor - Dr. John Shovic
      \par}
    
     \vspace{1cm} 
    
    \begin{figure}[h]
      \centering
      \includegraphics[width=0.6\linewidth]{assets/uislogan.png}
    \end{figure} 
  
		\vfill		
	\end{titlepage}

	\tableofcontents
	\newpage
	
	\section{Introduction}
	
		\subsection{Goal}
		Autonomous navigation and sampling of lake sediment in North Idaho lakes up to 1200 feet in depth. Project is being operated in cooperation with Gizmo and CDA Maker Space
	
		\subsection{Project Summary}
		The western region of the United States has been home to expensive mining operations, and there is an abundance of abandoned hard rock mines that fill this landscape. These mines contain dangerous toxins that contaminate nearby soils and water. Thus a large portion of headwater streams in the Western United States have been effected by this. These toxins and metals can be transmitted to lake basins, which can become repositories for a large quantity of sediment associated metals. Coeur d'Alene lake in Northern Idaho, where silver and lead mining in the South fork of Coeur d'Alene River has carried heavy metal contaminated sediments to the lakebed. Thus in Coeur d'Alene lake this is a prominent issue, and while these issues are apparent there, it is not just this lake that has this problem. Legacy Contamination within river beads is a problem on an international scale. Computer models also do not provide an accurate characterization of the transport of contamination, a more direct sensing of contamination is required. This is where the research for this project comes into play, to develop the Coeur d'Alene Catfish, which is an autonomous robotic drone that is capable of reading water quality information from Coeur d'Alene lake, and other deep water lakes. The end result then should be the development (key word is development, not completion) of a submarine that can provide autonomous deployment within Coeur d'Alene lake and other deep water lakes. Reservoirs are also desired as well, which can be fairly difficult to navigate due to the problematic kinetic nature of the environment. This in turn allows public to the supervision of water bodies in local communities as well. The results from such surveys conducted by the drone will be shared with other interested stakeholders, this will include the Idaho Water Department of Environmental Quality (IDEQ) and Coeur d'Alene Tribe.\\
		The research focuses on creating and/or starting beginning development into the infrastructure of an autonomous submarine (and sensor technologies). Thus the long term goal is a fully functional autonomous submarine that can collect water quality data in deep-water lakes and reservoirs. The short term goal is to develop the "CDA Catfish", submarine that can perform underwater surveys by continuously sampling a variety of water quality variables (oxygen, pH, temperature, etc).\\
		Interested parties/sponsors for pursuing this research include the Idaho Water Resource Research Institute, the United States Geologic Survey, and the University of Idaho. The research is also in cooperation with Gizmo, Coeur d'Alene Maker Space.\\
		Documentation and code for the project can be found at \url{https://github.com/TimetoPretend54/The-Coeur-d-Alene-Catfish}
		
		\subsection{Document Purpose}
			This document is a research portfolio for the University of Idaho Research Project for the Computer Science Department pertaining to underwater autonomous navigation. The purpose of this documents is to outline the methodology, design, decisions, evaluations, and to keep a record of this project. It defines the terms used, outlines the scope of the project, details of specific design choices, meeting minutes, project learning, design goals, specification and constraints, system diagrams, analysis of alternatives, engineering modeling, manufacturing/assembly plan, experimental design, data analysis, balance sheet, and other items.
		
		\subsection{Definition of Terms}
			\begin{itemize}
				\item \textbf{G2X} - A remotely operated underwater vehicle developed by Gizmo-cda, a makerspace located in Coeur d'Alene Idaho. The vehicle itself operates with 5 thrusters, with a Raspberry Pi (Model 3) as the SBC (Single Board Controller), alongside additional components. Components are located within the navigation module of the G2X. The vehicle can be operated via a controller (using the Pygame Python library and a Playstation 4 Controller), requiring a tether (handling gear) to provide the input, or can be run autonomously with a program, running on the G2X's Raspberry P (no tether), or a host computer (tether required). Operational Time is 6 hours on full charge, thrusters and raspberry pi operate on separate batteries, 8 hours for Pi, 6 hours for thrusters.Software for operating the software is provided courtesy of Gizmo, located at \url{https://github.com/gizmo-cda/g2x-submarine-v2}
				\item \textbf{Navigation Module} - Module located within the G2X that houses the the primary components of the vehicle, components are a Raspberry Pi 3, PiSenseHat (sensor module that includes internal pressure, temperature, compass, accelerometer, and gyroscope readings), a PWM Hat (for handling the PWM thrusters), an Arduino Nano for reading voltage, 5 Electronic Speed Controllers (ESC), a power regulator, pressure, temperature sensor for external treadings, and a Pi Camera V2 NoIR (video streaming).
				\item \textbf{ROS} - Robot Operating System. It is a robotics middle collection of software frameworks for robot software development. ROS Node is just an executable file within a ROS package, utilizing the ROS client library to communicate with other nodes\\ (https://en.wikipedia.org/wiki/Robot\textunderscore Operating\textunderscore System)
				\item \textbf{Raspberry Pi} - Single Board Computer (SBC) that is a small, compact (about the size of credit card), and affordable computer that promotes a wide variety of applications. The Operating System (OS) that runs on the Pi is Raspberian, which is based on the Debian operating system (Linux Distribution).\\ (https://en.wikipedia.org/wiki/Raspberry\textunderscore Pi)	
				\item \textbf{Arduino} - Open source computer hardware and software company, project, and user community that designs and manufactures 		single-board microcontrollers and microcontroller kits for building digital devices and interactive objects that can sense and control objects in the physical world\\ (https://en.wikipedia.org/wiki/Arduino)		
				\item \textbf{Handling Gear} - Gear and metal frame for lowering the G2X into the water (via a wench), a fiber optic tether that provides video streaming, alongside controller output (Playstation 4 Controller). The gear provides simple switches for retracting and extracting both the wench and tether. Powered by a 12V car battery, and operational for more than 36+ hours. The handling gear has the communications module attached.
				\item \textbf{Communications Module} - A module that houses an additional raspberry pi, the module is a network interface for transferring data between the G2X and a host computer. It contains the Pi, external pressure/temperature sensor, 1 GB fiber optic media converter, 4-port GB switch, Wide-angle camera, real time clock, power supply (12V from handling gear).
				\item \textbf{TurtleBot} - TurtleBot is a low-cost, personal robot kit with open-source software. With TurtleBot, you’ll can build a robot that can drive around, see in 3D, and create applications.The TurtleBot kit consists of a mobile base, 2D/3D distance sensor, laptop computer or SBC(Raspberry Pi), and the TurtleBot mounting hardware kit.\\
				(https://wiki.ros.org/Robots/TurtleBot)
				\item \textbf{Octomap} - An efficient probabilistic 3D mapping framework based on octrees. It is a ROS Library that implements a 3D occupancy grid mapping approach providing data structures and mapping algorithms in C++. The map implementation is based on an octree (tree data structure in which each internal node has exactly eight children). \\
				(https://wiki.ros.org/octomap)
				\item \textbf{Moveit!} - A motion planning framework.It is a software framework for mobile manipulation, including utilizing advances in motion planning, manipulation, 3D perception, kinematics, control and navigation. Thus providing a simple platform for developing robotics applications.\\
				(https://moveit.ros.org/)
				\item \textbf{SLAM} - Simultaneous localization and mapping. SLAM is a computational problem of constructing or updating a map of an unknown environment while simultaneously keeping track of an agent's location within it. There are several algorithms known for solving it (at least approximately).\\
				(https://en.wikipedia.org/wiki/Simultaneous\textunderscore localization\textunderscore and\textunderscore mapping)
			\end{itemize}
		
		\subsubsection{ROS - Additional Info}
		The Robot Operating System contains many essential features, including: hardware abstraction, low-level device control, message-passing between processes, and package management. ROS nodes are written in either C++ or Python. ROS is primarily supported on Ubuntu Linux, with other major OS being designated as "Experimental". The underwater research project will be using ROS Kinetic Kame (2016) as its ROS version, due to its supported features and stability. 
		
		\newpage
	
	\section{Meetings and Minutes}
	Weekly action items and summaries of progress made are detailed below. Furthermore, subsections discuss what was helpful and what was not during these meetings. Discussion of attendance and participation, as well as contribution and discussion topics are discussed below.
	
		\subsection{5/24/2018 Team Meeting 1 Notes}
		
			\noindent
			Project Scope Breakdown (Dr. Shovic, Adrian, Samantha started 9:00 am):
			
			\noindent
			\begin{itemize}
				\item Submarine Delivery 
				\begin{itemize}
					\item Submarine in Gizmo office
					\item Will be delivered tomorrow
					\begin{itemize}
						\item Keep submarine in lab
					\end{itemize}
				\end{itemize}
				\item Notebook
				\begin{itemize}
					\item Samantha and Adrian need one
				\end{itemize}
				\item Research
				\begin{itemize}
					\item Need to write a research paper
					\item At end of summer
					\item All teammates
				\end{itemize}
				\item Logbook
				\begin{itemize}
					\item Need to write about everything done to submarine
					\item Clearly designed and obvious
				\end{itemize}
				\item Three Main Items to Research
				\begin{enumerate}
					\item Do Field Trials
					\begin{itemize}
						\item June = testrun of Gizmo Submarine
						\item July = team software running on it
					\end{itemize}
					\item Get Team's Software Running on Submarine
					\item Fully Document Submarine
					\begin{itemize}
						\item Research Paper
					\end{itemize}
				\end{enumerate}
				\item Goals
				\begin{itemize}
					\item Adrian - head of project
					\begin{itemize}
						\item Allocate resources to make goals
					\end{itemize}
					\item Get current toolchain working (Gizmo)
				\end{itemize}
				\item Documentation
				\begin{itemize}
					\item Essential for project
					\item Set up our own Google Drive for University of Idaho Catfish
					\item Blank documents for what we know we need
					\item GitHub Repo
					\item Research portfolio
					\item Operations manual
					\item Need good documentation as Submarine may be senior project for next academic year
					\item Document Sub-systems of submarine
				\end{itemize}
				\item Cooperate
				\begin{itemize}
					\item University of South Dakota has expressed interest in similar research
				\end{itemize}
				\item Events
				\begin{itemize}
					\item Bayview visit, Dr. Shovic will be visiting next week on Monday
					\item 26th of July
					\begin{itemize}
						\item Vice President of Research
						\begin{itemize}
							\item Presentation for her
						\end{itemize}
						\item Schedule when Field Tests Occur
						\begin{itemize}
							\item July 24th = Trials
						\end{itemize}
					\end{itemize}
				\end{itemize}
				\item Architecture
				\begin{itemize}
					\item Graphic with items labled on it
					\item Two types:
					\begin{enumerate}
						\item With downward camera
						\item Without downward camera
					\end{enumerate}
				\end{itemize}
				\item Trials
				\begin{enumerate}
					\item June-Week of the 25th
					\begin{itemize}
						\item Provide more for operations manual and get experience
					\end{itemize}
					\item July - Late July (25th)
					\begin{itemize}
						\item Alternative Date: August 8th
						\item Also still tethered
					\end{itemize}
					\item So what for 2nd Field Test?
					\begin{itemize}
						\item Some amount of autonomous navigation
						\item No GPS system at that point
					\end{itemize}
				\end{enumerate}
				\item Submarine
				\begin{itemize}
					\item Sensors, communicator, actuator
					\item Sensor
					\begin{itemize}
						\item Camera in front
						\begin{itemize}
							\item Edited a lot, multiple resolutions
							\item Have 2TB disk on hand
						\end{itemize}
						\item Compass/Accelerometer
						\begin{itemize}
							\item Next to useless
							\item Due to metal casing
						\end{itemize}
						\item Sonar
						\begin{itemize}
							\item Need to look at research papers
							\item Four sonars? (up, down, left, right)
							\item Up = GPS station keep (water drone with GPS receiver)
							\item Listen to sonar and center itself
						\end{itemize}
						\item Acoustic Modem (ACM)
						\begin{itemize}
							\item Status data up
							\item GPS data down
							\item 30 k/bits a second or 90 k/bits
							\item Might be able to use as navigational thing
						\end{itemize}
						\item Before 2nd Trial
						\begin{itemize}
							\item 100 ft range with these sonars - Can we test?
							\item Large water through?
							\item Not large enough or submarine likely
							\item But can test sonars at least
						\end{itemize}
					\end{itemize}
				\end{itemize}
				\item Pod
				
			\end{itemize}

		\newpage

	\section{Project Learning}
	Technologies used to solve problems are described below. Further discussion of these technologies are left in each section's subsections.

		\newpage
		
	\section{Design Goals}
	Client needs and project goals are discussed below. A Timeline for these is also included. Discussion of revision of goals, and addition of any new goals is also discussed below.
	
		\newpage	
	
	\section{Specifications and Constraints}
	Discussion of client interviews, pictures, measurements, etc. are provided below. Design specifications and constraints are also presented. Reasoning for any constraints is also mentioned.
	
		\newpage
						
	\section{System Diagrams}
	Discussion of symbols used, the diagrams themselves, and the software used for the diagrams is discussed below.
	
		\newpage				
					
	\section{Analysis of Alternatives}
	Discussion of possible alternatives and why some alternatives are better is described below. These topics include: safety, moving parts, cost, durability, compatibility, and reliability.
	
		\newpage
	
	\section{Engineering Model}
	Discussion of the physical, chemical, and biological system modeling. Also discusses modeling criteria, expected accuracy, and pitfalls. Section of modeling software used is present, as well as data needed and how the data was obtained. Lastly a validation scheme for the model is shown.
	
		\newpage
	
	\section{Manufacturing/Assembly Plan}
	Discussion of the fabrication need, a flowchart of process oriented projects, a bill of materials, and the estimated manufacturer and delivery time is discussed below.
	
		\newpage
	
	\section{Experimental Design}
	The characterization of the purpose of the experiment, model validation, data gaps, and performance measurement are discussed below. Also the details on documentation, instrumentation, and measurements are also described.
		
		\newpage	
	
	\section{Data Analysis}
	Documentation on statistical tools used, accuracy of data, and experiments shown below. Discussion on confidence is results also discussed below.
		
		\newpage
	
	\section{Balance Sheet}
	Discussion on initial budget, estimated cost for materials, components, labor, and spending plan are all described below.
		
		\newpage				
						
	\section{Other Items}
	File management, archiving, documenting any issues, reports of accidents/incidents/near misses/precautions are described below.


\end{document}
