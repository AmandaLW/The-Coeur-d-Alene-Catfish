\documentclass[12pt]{article}
\usepackage[paperwidth=8.5in,paperheight=11in,margin=1in]{geometry}
\usepackage{float}
\usepackage{lipsum}
\usepackage{parskip}
\usepackage{bbding}
\usepackage{amssymb}
\usepackage{titlesec} 
\usepackage{graphicx}
\usepackage{hyperref}
\usepackage{setspace}
\usepackage[normalem]{ulem}
\usepackage[section]{placeins}
\usepackage[toc,page]{appendix}
\newcounter{subsubsubsection}[subsubsection]
\newcommand{\tline}{\hspace{-2.3pt}$\bullet$ \hspace{5pt}}
\hypersetup{colorlinks=true, linkcolor=black, urlcolor=blue}
\setlength{\parindent}{15pt} % Indent paragraphs (automatically)
\usepackage{pdfpages, caption}


\makeatother
\makeatletter
\setlength{\@fptop}{0pt}

\newcommand\tab[1][1cm]{\hspace*{#1}}

\definecolor{myRed}{RGB}{248, 0, 0}
\definecolor{myGreen}{RGB}{0, 208, 0} %full green too light -P
\definecolor{myBlue}{RGB}{0, 0, 248}

\begin{document}
	\begin{titlepage}
		\centering	
%		\vspace{.25cm}
    
    \begin{figure}[h]
      \centering
%      \includegraphics[width=0.45\linewidth]{icon for team goes here}
    \end{figure} 
  
  {\huge\bfseries The Coeur d'Alene Catfish: \\ Research Portfolio\par}
    
    \title{}
    \date{\vspace{-5ex}} %blank date -P
    \author{%
    	\makebox[.3\linewidth]{\Large\itshape Adrian Beehner}\\Team Manager\\
    	\and \makebox[.3\linewidth]{\Large\itshape Samantha Freitas}\\Designer\\
    }
    \let\newpage\relax\maketitle %don't make a new page -P
    \maketitle		
    
    \vspace{4cm} 
    
    {\scshape\Large 
      CS Department Research: May 2017 - August 2018 \\
      Autonomous Robotic Submarine Project\\ 
      Sponsors - Idaho Water Resources Research Institue, United States Geologic Survey, University of Idaho \\
      Supervisor - Dr. John Shovic
      \par}
    
     \vspace{4cm} 
    
    \begin{figure}[h]
      \centering
      \includegraphics[width=0.7\linewidth]{assets/uislogan.png}
    \end{figure} 
  
		\vfill		
	\end{titlepage}

	\tableofcontents
	\newpage
	
	\section{Introduction}
	
		\subsection{Goal}
		Autonomous naviagation and sampling of lake sediment in North Idaho lakes up to 1200 feet in depth. Project is being operated in cooperation with Gizmo and CDA Maker Space
	
		\subsection{Project Summary - JUST EXAMPLE}
		The University of Idaho has, for several years, done various projects involving the Tower of Lights Show and equipping the marching band with light-up glasses. The current "TowerLights" product involves LED-based light bars that are placed in front of front-facing widows of a large buildling (Theophilus Tower) and are then illuminated to play animations alongside/synchronously with music. The goal is to enhance the current "TowerLights" product. The current implementation of this product uses the ethernet wiring system in the building to control the LEDs. The goal of the project described in this document is to convert this part of the system to a wireless operation. This in turn requires the development of a wireless module that would be attached to each of the light bars. Thus this module has to sleep and wake up, as well as respond to wireless signals from a computer, and since it's wireless, these modules will need to be battery powered. Battery power must also be conserved by staying in the sleep state until needed. The purpose of this enhancement is to provide a certain level of portability to have "TowerLights" at other locations. \\
		The product will give the user the ability to run a program that reads in .tan files and .wav files, have this program communicate with a XBee Wireless module on an Arduino that is attached to a computer via USB, then communicate wirelessly with each Arduino receiver, that is battery powered. Each of these Arduino receivers are attached to an LED board, that will then communicate with each LED on that board through wired communication from the Arduino (same one that holds the receiver) to the LEDs. The program that broadcasts the shows will be available for OSX, Windows, and Linux based operating systems.\\
		This documentation lives at \url{https://github.com/YupHio/LEaD_Design/tree/master/Doc/TeamPortfolio_LEaD_Design.tex} \\
		The code for the project can be found at \url{https://github.com/YupHio/LEaD_Design/tree/master/Code}
		
		\subsection{Document Purpose - JUST EXAMPLE}
	 		This document is a team portfolio for the Fall 2017-Spring 2018 CS 480/481: Senior Capstone Design project at the University of Idaho. The purpose of this document is to outline the methodology, design, and keep a record of this project. It defines terms used, outlines the scope of the project, details specific design choices, meeting minutes, project learning, design goals, specification and constraints, system diagrams, analysis of alternatives, engineering modeling, manufacturing/assembly plan, experimental design, data analysis, balance sheet, and other items.
		
		\subsection{Definition of Terms - JUST EXAMPLE}
			\begin{itemize}
				\item \textbf{Arduino} - open source computer hardware and software company, project, and user community that designs and manufactures 		single-board microcontrollers and microcontroller kits for building digital devices and interactive objects that can sense and control objects in the physical world\\ (https://en.wikipedia.org/wiki/Arduino)
				\item \textbf{Arduino Shield} - Shields are boards that can be plugged on top of the Arduino PCB extending its capabilities. The different shields follow the same philosophy as the original toolkit: they are easy to mount, and cheap to produce.\\ (https://www.arduino.cc/en/Main/ArduinoShields)
				\item \textbf{Xbee} - The Arduino Xbee shield allows multiple Arduino boards to communicate wirelessly over distances up to 100 feet (indoors) or 300 feet (outdoors) using the Maxstream Xbee Zigbee module.\\ (https://www.arduino.cc/en/Main/ArduinoShields)		
			\end{itemize}
		
		\subsubsection{Arduino IDE}
		The Arduino Integrated Development Environment - or Arduino Software (IDE) - contains a text editor for writing code, a message area, a text console, a toolbar with buttons for common functions and a series of menus. It connects to the Arduino and Genuino hardware to upload programs and communicate with them. \url{https://www.arduino.cc/en/Main/Software}
		
		\newpage
	
	\section{Meetings and Minutes}
	Weekly action items and summaries of progress made are detailed below. Furthermore, subsections discuss what was helpful and what was not during these meetings. Discussion of attendance and participation, as well as contribution and discussion topics are discussed below.

		\newpage

	\section{Project Learning}
	Technologies used to solve problems are described below. Further discussion of these technologies are left in each section's subsections.

		\newpage
		
	\section{Design Goals}
	Client needs and project goals are discussed below. A Timeline for these is also included. Discussion of revision of goals, and addition of any new goals is also discussed below.
	
		\newpage	
	
	\section{Specifications and Constraints}
	Discussion of client interviews, pictures, measurements, etc. are provided below. Design specifications and constraints are also presented. Reasoning for any constraints is also mentioned.
	
		\newpage
						
	\section{System Diagrams}
	Discussion of symbols used, the diagrams themselves, and the software used for the diagrams is discussed below.
	
		\newpage				
					
	\section{Analysis of Alternatives}
	Discussion of possible alternatives and why some alternatives are better is described below. These topics include: safety, moving parts, cost, durability, compatibility, and reliability.
	
		\newpage
	
	\section{Engineering Model}
	Discussion of the physical, chemical, and biological system modeling. Also discusses modeling criteria, expected accuracy, and pitfalls. Section of modeling software used is present, as well as data needed and how the data was obtained. Lastly a validation scheme for the model is shown.
	
		\newpage
	
	\section{Manufacturing/Assembly Plan}
	Discussion of the fabrication need, a flowchart of process oriented projects, a bill of materials, and the estimated manufacturer and delivery time is discussed below.
	
		\newpage
	
	\section{Experimental Design}
	The characterization of the purpose of the experiment, model validation, data gaps, and performance measurement are discussed below. Also the details on documentation, instrumentation, and measurements are also described.
		
		\newpage	
	
	\section{Data Analysis}
	Documentation on statistical tools used, accuracy of data, and experiments shown below. Discussion on confidence is results also discussed below.
		
		\newpage
	
	\section{Balance Sheet}
	Discussion on initial budget, estimated cost for materials, components, labor, and spending plan are all described below.
		
		\newpage				
						
	\section{Other Items}
	File management, archiving, documenting any issues, reports of accidents/incidents/near misses/precautions are described below.


\end{document}
