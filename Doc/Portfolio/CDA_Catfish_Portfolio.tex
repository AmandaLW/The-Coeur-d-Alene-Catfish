\documentclass[12pt]{article}
\usepackage[paperwidth=8.5in,paperheight=11in,margin=1in]{geometry}
\usepackage{float}
\usepackage{lipsum}
\usepackage{parskip}
\usepackage{bbding}
\usepackage{amssymb}
\usepackage{titlesec} 
\usepackage{graphicx}
\usepackage{hyperref}
\usepackage{setspace}
\usepackage[normalem]{ulem}
\usepackage[section]{placeins}
\usepackage[toc,page]{appendix}
\newcounter{subsubsubsection}[subsubsection]
\newcommand{\tline}{\hspace{-2.3pt}$\bullet$ \hspace{5pt}}
\hypersetup{colorlinks=true, linkcolor=black, urlcolor=blue}
\setlength{\parindent}{15pt} % Indent paragraphs (automatically)
\usepackage{pdfpages, caption}


\makeatother
\makeatletter
\setlength{\@fptop}{0pt}

\newcommand\tab[1][1cm]{\hspace*{#1}}

\definecolor{myRed}{RGB}{248, 0, 0}
\definecolor{myGreen}{RGB}{0, 208, 0} %full green too light -P
\definecolor{myBlue}{RGB}{0, 0, 248}

\begin{document}
	\begin{titlepage}
		\centering	
%		\vspace{.25cm}
    
    \begin{figure}[h]
      \centering
      \includegraphics[width=0.38\linewidth]{assets/New_Logov1.png}
    \end{figure} 
  
  {\huge\bfseries The Coeur d'Alene Catfish: \\ Research Portfolio\par}
    
    \title{}
    \date{\vspace{-5ex}} %blank date -P
    \author{%
    	\makebox[.3\linewidth]{\Large\itshape Adrian Beehner}\\Team Manager\\
    	\and \makebox[.3\linewidth]{\Large\itshape Samantha Freitas}\\Designer\\
    }
    \let\newpage\relax\maketitle %don't make a new page -P
    \maketitle		
    
    \vspace{1cm} 
    
    {\scshape\Large 
      May 2018 - August 2018 \\
      Autonomous Robotic Submarine Project\\ 
      Sponsors - Idaho Water Resources Research Institue, United States Geologic Survey, UofI \\
      Supervisor - Dr. John Shovic
      \par}
    
     \vspace{1cm} 
    
    \begin{figure}[h]
      \centering
      \includegraphics[width=0.6\linewidth]{assets/uislogan.png}
    \end{figure} 
  
		\vfill		
	\end{titlepage}

	\tableofcontents
	\newpage
	
	\section{Introduction}
	
		\subsection{Goal}
		Autonomous naviagation and sampling of lake sediment in North Idaho lakes up to 1200 feet in depth. Project is being operated in cooperation with Gizmo and CDA Maker Space
	
		\subsection{Project Summary}
		The western region of the United States has been home to expensive mining operations, and there is an abundance of abandoned hard rock mines that fill this landscape. These mines contain dangerous toxins that contaminate nearby soils and water. Thus a large portion of headwater streams in the Western United States have been effected by this. These toxins and metals can be transmited to lake basins, which can become repositories for a large quantity of sediment associated metals. Coeur d'Alene lake in Northern Idaho, where silver and lead mining in the South fork of Coeur d'Alene River has carried heavy metal contaminated sediments to the lakebed. Thus in Coeur d'Alene lake this is a prominent issue, and while these issues are apparent there, it is not just this lake that has this problem. Legacy Contamination within river beads is a problem on an international scale. Computer models also do not provide an accurate charactierization of the transport of contamination, a more direct sensing of contamination is required. This is where the research for this project comes into play, to develop the Coeur d'Alene Catfish, which is an autonomous robotic drone that is capable of reading water quality information from Coeur d'Alene lake, and other deep water lakes. The end result then should be the development (key word is development, not completion) of a submarine that can provide autonomous deployment within Coeur d'Alene lake and other deep water lakes. Reservoirs are also desired as well, which can be fairly difficult to navigate due to the problematic kinetic nature of the environment. This in turn allows public to the supervision of water bodies in local communities as well. The results from such surveys conducted by the drone will be shared with other interested stakeholders, this will include the Idaho Water Department of Evironmental Quality (IDEQ) and Coeur d'Alene Tribe.\\
		The research focuses on creating and/or starting beginning development into the infrastructure of an autonomous submarine (and sensor technologies). Thus the long term goal is a fully functional autonomous submarine that can collect water quality data in deep-water lakes and reservoirs. The short term goal is to develop the "CDA Catfish", submarine that can perform underwater surveys by continuously sampling a variety of water quality variables (oxygen, pH, temperature, etc).\\
		Interested parties/sponsors for pursuing this research include the Idahop Water Resource Research Instiute, the United States Geologic Survey, and the University of Idaho. The research is also in cooperation with Gizmo, Coeur d'Alene Maker Space.\\
		Documentation and code for the project can be found at \url{https://github.com/TimetoPretend54/The-Coeur-d-Alene-Catfish}
		
		\subsection{Document Purpose}
			This document is a research portfolio for the University of Idaho Research Project for the Computer Science Department pertaining to underwater autonomous navigation. The purpose of this documents is to outline the methodoloy, design, decisions, evalulations, and to keep a record of this project. It defines the terms used, outlines the scope of the project, details of specific design choices, meeting minutes, project learning, design goals, specification and contraints, system diagrams, analysis of alternatives, engineering modeling, manufacturing/assemply plan, experimental design, data analysis, balance sheet, and other items.
		
		\subsection{Definition of Terms}
			\begin{itemize}
				\item \textbf{G2X} - A remotely operated underwater vehicle devloped by Gizmo-cda, a makerspace located in Coeur d'Alene Idaho. The vehicle itself operates with 5 thrusters, with a Raspberry Pi (Model 3) as the SBC (Single Board Controller), alongside additional components. Components are located within the navigation module of the G2X. The vehicle can be operated via a controller (using the Pygame Python library and a Playstation 4 Controller), requiring a tether (handling gear) to provide the input, or can be run autonomously with a program, running on the G2X's Raspberry P (no tether), or a host computer (tether required). Opertaional Time is 6 hours on full charge, thrusters and raspberry pi operate on separate batteries, 8 hours for Pi, 6 hours for thrusters.Software for operating the software is provided courtesy of Gizmo, located at \url{https://github.com/gizmo-cda/g2x-submarine-v2}
				\item \textbf{Navigation Module} - Module located within the G2X that houses the the primary components of the vehicle, components are a Raspberry Pi 3, PiSenseHat (sensor module that includes internal pressue, temperature, compass, accelerometer, and gyroscope readings), a PWM Hat (for handling the PWM thrusters), an Arduino Nano for reading voltage, 5 Electronic Speed Controllers (ESC), a power regulator, pressure, termperature sensor for external treadings, and a Pi Camera V2 NoIR (video streaming).
				\item \textbf{ROS} - Robot Operating System. It is a robotics middle collection of software frameworks for robot software development. ROS Node is just an executable file within a ROS package, utilizing the ROS client library to communicte with other nodes\\ (https://en.wikipedia.org/wiki/Robot\textunderscore Operating\textunderscore System)
				\item \textbf{Raspberry Pi} - Single Board Compiter (SBC) that is a small, compact (about the size of credit card), and affordable computer that promotes a wide variety of applications. The Operating System (OS) that runs on the Pi is Raspberian, which is based on the Debian operating system (Linux Distrubution).\\ (https://en.wikipedia.org/wiki/Raspberry\textunderscore Pi)	
				\item \textbf{Arduino} - Open source computer hardware and software company, project, and user community that designs and manufactures 		single-board microcontrollers and microcontroller kits for building digital devices and interactive objects that can sense and control objects in the physical world\\ (https://en.wikipedia.org/wiki/Arduino)		
				\item \textbf{Handling Gear} - Gear and metal frame for lowering the G2X into the water (via a wench), a fiber optic tether that provides video streaming, alongside controller ouput (Playstation 4 Contoller). The gear provides simple switches for retracting and extracting both the wench and tether. Powered by a 12V car battery, and operational for more than 36+ hours (on full charge). The handlign gear has the communications module attached.
				\item \textbf{Communications Module} - A module that houses an additional raspberry pi, the module is a network interface for trasnferring data between the G2X and a host computer. It conatins the Pi, external pressure/temperature sensor, 1 GB fiberoptic media converter, 4-port GB switch, Wide-agngle camera, realtime clock, powersupply (12V from handling gear).
				\item \textbf{TurtleBot3} - A module that houses an additional raspberry pi, the module is a network interface for trasnferring data between the G2X and a host computer. It conatins the Pi, external pressure/temperature sensor, 1 GB fiberoptic media converter, 4-port GB switch, Wide-agngle camera, realtime clock, powersupply (12V from handling gear).
				\item \textbf{Octomap} - A module that houses an additional raspberry pi, the module is a network interface for trasnferring data between the G2X and a host computer. It conatins the Pi, external pressure/temperature sensor, 1 GB fiberoptic media converter, 4-port GB switch, Wide-agngle camera, realtime clock, powersupply (12V from handling gear).
				\item \textbf{Moveit!} - A module that houses an additional raspberry pi, the module is a network interface for trasnferring data between the G2X and a host computer. It conatins the Pi, external pressure/temperature sensor, 1 GB fiberoptic media converter, 4-port GB switch, Wide-agngle camera, realtime clock, powersupply (12V from handling gear).
				\item \textbf{SLAM} - A module that houses an additional raspberry pi, the module is a network interface for trasnferring data between the G2X and a host computer. It conatins the Pi, external pressure/temperature sensor, 1 GB fiberoptic media converter, 4-port GB switch, Wide-agngle camera, realtime clock, powersupply (12V from handling gear).
			\end{itemize}
		
		\subsubsection{ROS - Additional Info}
		The Robot Operating System contains many essential features, including: hardware abstraction, low-level device control, message-passing between processes, and package managment. ROS nodes are written in either C++ or Pyhton. ROS is primarily supportd on Ubuntu Linux, with other major OS being designated as "Experimental". ROS Versions from 2014 and up include: Indigo Igloo (2014), Jade Turtle (2015), Kinetic Kame (2016), Lunar Loggerhead (2017), and Melodic Morenia (2018). The underwater research project will be using ROS Kinetic Kame as its ROS version, due to its supported features and stability. 
		
		\newpage
	
	\section{Meetings and Minutes}
	Weekly action items and summaries of progress made are detailed below. Furthermore, subsections discuss what was helpful and what was not during these meetings. Discussion of attendance and participation, as well as contribution and discussion topics are discussed below.

		\newpage

	\section{Project Learning}
	Technologies used to solve problems are described below. Further discussion of these technologies are left in each section's subsections.

		\newpage
		
	\section{Design Goals}
	Client needs and project goals are discussed below. A Timeline for these is also included. Discussion of revision of goals, and addition of any new goals is also discussed below.
	
		\newpage	
	
	\section{Specifications and Constraints}
	Discussion of client interviews, pictures, measurements, etc. are provided below. Design specifications and constraints are also presented. Reasoning for any constraints is also mentioned.
	
		\newpage
						
	\section{System Diagrams}
	Discussion of symbols used, the diagrams themselves, and the software used for the diagrams is discussed below.
	
		\newpage				
					
	\section{Analysis of Alternatives}
	Discussion of possible alternatives and why some alternatives are better is described below. These topics include: safety, moving parts, cost, durability, compatibility, and reliability.
	
		\newpage
	
	\section{Engineering Model}
	Discussion of the physical, chemical, and biological system modeling. Also discusses modeling criteria, expected accuracy, and pitfalls. Section of modeling software used is present, as well as data needed and how the data was obtained. Lastly a validation scheme for the model is shown.
	
		\newpage
	
	\section{Manufacturing/Assembly Plan}
	Discussion of the fabrication need, a flowchart of process oriented projects, a bill of materials, and the estimated manufacturer and delivery time is discussed below.
	
		\newpage
	
	\section{Experimental Design}
	The characterization of the purpose of the experiment, model validation, data gaps, and performance measurement are discussed below. Also the details on documentation, instrumentation, and measurements are also described.
		
		\newpage	
	
	\section{Data Analysis}
	Documentation on statistical tools used, accuracy of data, and experiments shown below. Discussion on confidence is results also discussed below.
		
		\newpage
	
	\section{Balance Sheet}
	Discussion on initial budget, estimated cost for materials, components, labor, and spending plan are all described below.
		
		\newpage				
						
	\section{Other Items}
	File management, archiving, documenting any issues, reports of accidents/incidents/near misses/precautions are described below.


\end{document}
