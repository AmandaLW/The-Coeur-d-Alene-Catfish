\documentclass[12pt]{article}
\usepackage[paperwidth=8.5in,paperheight=11in,margin=1in]{geometry}
\usepackage{float}
\usepackage{lipsum}
\usepackage{parskip}
\usepackage{bbding}
\usepackage{amssymb}
\usepackage{titlesec} 
\usepackage{graphicx}
\usepackage{hyperref}
\usepackage{setspace}
\usepackage[normalem]{ulem}
\usepackage[section]{placeins}
\usepackage[toc,page]{appendix}
\newcounter{subsubsubsection}[subsubsection]
\newcommand{\tline}{\hspace{-2.3pt}$\bullet$ \hspace{5pt}}
\hypersetup{colorlinks=true, linkcolor=black, urlcolor=blue}
\setlength{\parindent}{15pt} % Indent paragraphs (automatically)
\usepackage{pdfpages, caption}


\makeatother
\makeatletter
\setlength{\@fptop}{0pt}

\newcommand\tab[1][1cm]{\hspace*{#1}}

\definecolor{myRed}{RGB}{248, 0, 0}
\definecolor{myGreen}{RGB}{0, 208, 0} %full green too light -P
\definecolor{myBlue}{RGB}{0, 0, 248}

\begin{document}
	\begin{titlepage}
		\centering	
%		\vspace{.25cm}
    
    \begin{figure}[h]
      \centering
%      \includegraphics[width=0.45\linewidth]{icon for team goes here}
    \end{figure} 
  
  {\huge\bfseries The Coeur d'Alene Catfish: \\ Research Portfolio\par}
    
    \title{}
    \date{\vspace{-5ex}} %blank date -P
    \author{%
    	\makebox[.3\linewidth]{\Large\itshape Adrian Beehner}\\Team Manager\\
    	\and \makebox[.3\linewidth]{\Large\itshape Samantha Freitas}\\Designer\\
    }
    \let\newpage\relax\maketitle %don't make a new page -P
    \maketitle		
    
    \vspace{4cm} 
    
    {\scshape\Large 
      CS Department Research: May 2017 - August 2018 \\
      Autonomous underwater drone for lake research \\ 
      Sponsor - IWWRI/Idaho EPSCoR \\
      Supervisor - Dr. John Shovic
      \par}
    
     \vspace{4cm} 
    
    \begin{figure}[h]
      \centering
      \includegraphics[width=0.7\linewidth]{assets/uislogan.png}
    \end{figure} 
  
		\vfill		
	\end{titlepage}

	\tableofcontents
	\newpage
	
	\section{Introduction}
	
		\subsection{Project Summary}
		The University of Idaho has, for several years, done various projects involving the Tower of Lights Show and equipping the marching band with light-up glasses. The current "TowerLights" product involves LED-based light bars that are placed in front of front-facing widows of a large buildling (Theophilus Tower) and are then illuminated to play animations alongside/synchronously with music. The goal is to enhance the current "TowerLights" product. The current implementation of this product uses the ethernet wiring system in the building to control the LEDs. The goal of the project described in this document is to convert this part of the system to a wireless operation. This in turn requires the development of a wireless module that would be attached to each of the light bars. Thus this module has to sleep and wake up, as well as respond to wireless signals from a computer, and since it's wireless, these modules will need to be battery powered. Battery power must also be conserved by staying in the sleep state until needed. The purpose of this enhancement is to provide a certain level of portability to have "TowerLights" at other locations. \\
		The product will give the user the ability to run a program that reads in .tan files and .wav files, have this program communicate with a XBee Wireless module on an Arduino that is attached to a computer via USB, then communicate wirelessly with each Arduino receiver, that is battery powered. Each of these Arduino receivers are attached to an LED board, that will then communicate with each LED on that board through wired communication from the Arduino (same one that holds the receiver) to the LEDs. The program that broadcasts the shows will be available for OSX, Windows, and Linux based operating systems.\\
		This documentation lives at \url{https://github.com/YupHio/LEaD_Design/tree/master/Doc/TeamPortfolio_LEaD_Design.tex} \\
		The code for the project can be found at \url{https://github.com/YupHio/LEaD_Design/tree/master/Code}
		
		\subsection{Document Purpose}
	 		This document is a team portfolio for the Fall 2017-Spring 2018 CS 480/481: Senior Capstone Design project at the University of Idaho. The purpose of this document is to outline the methodology, design, and keep a record of this project. It defines terms used, outlines the scope of the project, details specific design choices, meeting minutes, project learning, design goals, specification and constraints, system diagrams, analysis of alternatives, engineering modeling, manufacturing/assembly plan, experimental design, data analysis, balance sheet, and other items.
		
		\subsection{Definition of Terms}
			\begin{itemize}
				\item \textbf{Arduino} - open source computer hardware and software company, project, and user community that designs and manufactures 		single-board microcontrollers and microcontroller kits for building digital devices and interactive objects that can sense and control objects in the physical world\\ (https://en.wikipedia.org/wiki/Arduino)
				\item \textbf{Arduino Shield} - Shields are boards that can be plugged on top of the Arduino PCB extending its capabilities. The different shields follow the same philosophy as the original toolkit: they are easy to mount, and cheap to produce.\\ (https://www.arduino.cc/en/Main/ArduinoShields)
				\item \textbf{Xbee} - The Arduino Xbee shield allows multiple Arduino boards to communicate wirelessly over distances up to 100 feet (indoors) or 300 feet (outdoors) using the Maxstream Xbee Zigbee module.\\ (https://www.arduino.cc/en/Main/ArduinoShields)		
			\end{itemize}
		
		\subsubsection{Arduino IDE}
		The Arduino Integrated Development Environment - or Arduino Software (IDE) - contains a text editor for writing code, a message area, a text console, a toolbar with buttons for common functions and a series of menus. It connects to the Arduino and Genuino hardware to upload programs and communicate with them. \url{https://www.arduino.cc/en/Main/Software}
		
		\subsubsection{Pulse}
		 PulseAudio is a sound system for POSIX OSes, meaning that it is a proxy for your sound applications. It allows you to do advanced operations on your sound data as it passes between your application and your hardware. Things like transferring the audio to a different machine, changing the sample format or channel count and mixing several sounds into one are easily achieved using a sound server. \url{https://www.freedesktop.org/wiki/Software/PulseAudio/l}
		
	\newpage


\end{document}
