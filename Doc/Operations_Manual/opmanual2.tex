
\documentclass{tstextbook}

\begin{document}

\tsbook{ΕΙΣΑΓΩΓΗ ΣΤΗΝ \\ ΕΦΑΡΜΟΣΜΕΝΗ ΓΡΑΜΜΙΚΗ ΑΛΓΕΒΡΑ \\ \vspace{10mm} Διδακτικές Σημειώσεις}
       {Μανόλης Βάβαλης}
       {Cover Designer}
       {2017}
       {xxxxx}{xxx--xx--xxxx--xx--x}{0.0}
       {Publisher}
       {Βόλος}

%---------------------------------------------------------------------------
% Chapters
%---------------------------------------------------------------------------

%---------------------------------------------------------------------------
\chapter{Προθέρμανση}

\epigraph{Η Αλίκη έσκυψε κάτω από το φράχτη και χώθηκε στην τρύπα, δίχως να σκεφτεί με ποιον τρόπο θα έβγαινε από κει μέσα.}{\textit{Η Αλίκη στην Χώρα των Θαυμάτων, Λούις Κάρολ}}

\begin{summary}
  Στο πρώτο αυτό κεφάλαιο θα προσπαθήσουμε να αποκτήσουμε μια αρχική ιδέα αναφορικά με το γενικότερο αντικείμενο της Γραμμικής Άλγεβρας, τις βασικές έννοιες και τα γενικά χαρακτηριστικά της. Η ανάπτυξη του Κεφαλαίου αυτού είναι σχετικά ασαφής. Η ελπίδα είναι να διαμορφώσουμε μια εννοιλογική βαση πάνω στην οποία θα εργαστούμε στα επόμενα κεφάλαια. Mε την διαμόρφωση της εν λόγω βάσης θα συγκεκριμενοποιήσουμε τις αναφερόμενες έννοιες και θα κατανοήσουμε το γενικότερο πλαίσιο της Γραμμικής Άλγεβρας. 
%  Συνεπώς προτείνετε να ξαναεπισκεπτόμαστε το Κεφάλαιο αυτό στο τέλος καθενός απο τα επόμενα κεφάλαια. 
Επιπρόσθετα θα προσπαθήσουμε να πάρουμε μια πρώτη εικόνα τόσο της χρησιμότητας όσο και της σπουδαιότητας της Γραμμικής Άλγεβρας.
\end{summary}

\input{chapters/01_TiEinaiGA}

%---------------------------------------------------------------------------
\chapter{Η απαλοιφή του Γκάους}

\epigraph{Μπορείς να σκεφθείς κάποια μαθηματική έννοια η οποία είναι τόσο παλή στην διατύπωσή της που μπορεί να την διδάξεις με ευκολία στο Λύκειο, η οποία είναι τόσο χρήσιμη που χρησιμοποιείται χιλάδες ή και εκατομμύρια φορές κάθε μέρα, η οποία αποτελεί αντικείμενο μελέτης για τουλάχιστον 2000 χρόνια αλλά δεν έχει ακόμα κατανοηθεί πλήρως;}{\textit{Carl D. Meyer, NC State University}}

\begin{summary}
 Θα ξεκινήσουμε με τα ποιο εύκολα προς μελέτη προβλήματα τα οποία θα προσπαθήσουμε να επεκτείνουμε με διάφορους τρόπου και να τα γενικεύσουμε για τις ανάγκες μας. Θα καταλήξουμε με ένα απο τους πιό σημαντικούς επιστημονικούς αλγόριθμους την απαλοιφή του Γκάους. Η εν λόγω απαλοιφή πέρα απο την βοήθεια που θα μας προσφέρει στην πορεία μας για την θεωρητική και πρακτική μελέτη συστημάτων θα μας βοηθήσει και σε γενικότερα θέματα αλγοριθμηκά και όχι μόνον.
 
 Για την ευκολία μας στο κεφάλαιο αυτό θα περιοριστούμε σε τετραγωνικά συστήματα. Η ονοματολογία τους προκύπτει απο την τετραγωνική τους μορφή η οποία με την σειρά τους προκύπτει απο το γεγονός ότι αφορούν $n$ εξισώσει που εμπλέκουν $n$ αγνώστους, όπου $n$ ένας οποισδήποτε θετικός ακέραιος αριθμός, συχνά αρκετά μεγάλος.
\end{summary}

\input{chapters/02_Gauss}



%---------------------------------------------------------------------------
\chapter{Διανύσματα, Πίνακες και Πράξεις}

\epigraph{Unfortunately, no one can be told what the Matrix is. You have to see it for yourself.}{\textit{Morpheus to Neo, The Matrix}}

\begin{summary}
  \blindtext
\end{summary}

\section{First section}
\Blindtext

\section{Second section}
\Blindtext

\section{Third section}
\Blindtext

%---------------------------------------------------------------------------
% Bibliography
%---------------------------------------------------------------------------

% \addcontentsline{toc}{chapter}{\textcolor{tssteelblue}{Literature}}
% \printbibliography{}

%---------------------------------------------------------------------------
% Index
%---------------------------------------------------------------------------

\printindex

\end{document}

              